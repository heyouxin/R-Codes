\documentclass[]{article}
\usepackage{lmodern}
\usepackage{amssymb,amsmath}
\usepackage{ifxetex,ifluatex}
\usepackage{fixltx2e} % provides \textsubscript
\ifnum 0\ifxetex 1\fi\ifluatex 1\fi=0 % if pdftex
  \usepackage[T1]{fontenc}
  \usepackage[utf8]{inputenc}
\else % if luatex or xelatex
  \ifxetex
    \usepackage{mathspec}
  \else
    \usepackage{fontspec}
  \fi
  \defaultfontfeatures{Ligatures=TeX,Scale=MatchLowercase}
  \newcommand{\euro}{€}
\fi
% use upquote if available, for straight quotes in verbatim environments
\IfFileExists{upquote.sty}{\usepackage{upquote}}{}
% use microtype if available
\IfFileExists{microtype.sty}{%
\usepackage{microtype}
\UseMicrotypeSet[protrusion]{basicmath} % disable protrusion for tt fonts
}{}
\usepackage[margin=1in]{geometry}
\usepackage{hyperref}
\PassOptionsToPackage{usenames,dvipsnames}{color} % color is loaded by hyperref
\hypersetup{unicode=true,
            pdfborder={0 0 0},
            breaklinks=true}
\urlstyle{same}  % don't use monospace font for urls
\usepackage{color}
\usepackage{fancyvrb}
\newcommand{\VerbBar}{|}
\newcommand{\VERB}{\Verb[commandchars=\\\{\}]}
\DefineVerbatimEnvironment{Highlighting}{Verbatim}{commandchars=\\\{\}}
% Add ',fontsize=\small' for more characters per line
\usepackage{framed}
\definecolor{shadecolor}{RGB}{248,248,248}
\newenvironment{Shaded}{\begin{snugshade}}{\end{snugshade}}
\newcommand{\KeywordTok}[1]{\textcolor[rgb]{0.13,0.29,0.53}{\textbf{{#1}}}}
\newcommand{\DataTypeTok}[1]{\textcolor[rgb]{0.13,0.29,0.53}{{#1}}}
\newcommand{\DecValTok}[1]{\textcolor[rgb]{0.00,0.00,0.81}{{#1}}}
\newcommand{\BaseNTok}[1]{\textcolor[rgb]{0.00,0.00,0.81}{{#1}}}
\newcommand{\FloatTok}[1]{\textcolor[rgb]{0.00,0.00,0.81}{{#1}}}
\newcommand{\ConstantTok}[1]{\textcolor[rgb]{0.00,0.00,0.00}{{#1}}}
\newcommand{\CharTok}[1]{\textcolor[rgb]{0.31,0.60,0.02}{{#1}}}
\newcommand{\SpecialCharTok}[1]{\textcolor[rgb]{0.00,0.00,0.00}{{#1}}}
\newcommand{\StringTok}[1]{\textcolor[rgb]{0.31,0.60,0.02}{{#1}}}
\newcommand{\VerbatimStringTok}[1]{\textcolor[rgb]{0.31,0.60,0.02}{{#1}}}
\newcommand{\SpecialStringTok}[1]{\textcolor[rgb]{0.31,0.60,0.02}{{#1}}}
\newcommand{\ImportTok}[1]{{#1}}
\newcommand{\CommentTok}[1]{\textcolor[rgb]{0.56,0.35,0.01}{\textit{{#1}}}}
\newcommand{\DocumentationTok}[1]{\textcolor[rgb]{0.56,0.35,0.01}{\textbf{\textit{{#1}}}}}
\newcommand{\AnnotationTok}[1]{\textcolor[rgb]{0.56,0.35,0.01}{\textbf{\textit{{#1}}}}}
\newcommand{\CommentVarTok}[1]{\textcolor[rgb]{0.56,0.35,0.01}{\textbf{\textit{{#1}}}}}
\newcommand{\OtherTok}[1]{\textcolor[rgb]{0.56,0.35,0.01}{{#1}}}
\newcommand{\FunctionTok}[1]{\textcolor[rgb]{0.00,0.00,0.00}{{#1}}}
\newcommand{\VariableTok}[1]{\textcolor[rgb]{0.00,0.00,0.00}{{#1}}}
\newcommand{\ControlFlowTok}[1]{\textcolor[rgb]{0.13,0.29,0.53}{\textbf{{#1}}}}
\newcommand{\OperatorTok}[1]{\textcolor[rgb]{0.81,0.36,0.00}{\textbf{{#1}}}}
\newcommand{\BuiltInTok}[1]{{#1}}
\newcommand{\ExtensionTok}[1]{{#1}}
\newcommand{\PreprocessorTok}[1]{\textcolor[rgb]{0.56,0.35,0.01}{\textit{{#1}}}}
\newcommand{\AttributeTok}[1]{\textcolor[rgb]{0.77,0.63,0.00}{{#1}}}
\newcommand{\RegionMarkerTok}[1]{{#1}}
\newcommand{\InformationTok}[1]{\textcolor[rgb]{0.56,0.35,0.01}{\textbf{\textit{{#1}}}}}
\newcommand{\WarningTok}[1]{\textcolor[rgb]{0.56,0.35,0.01}{\textbf{\textit{{#1}}}}}
\newcommand{\AlertTok}[1]{\textcolor[rgb]{0.94,0.16,0.16}{{#1}}}
\newcommand{\ErrorTok}[1]{\textcolor[rgb]{0.64,0.00,0.00}{\textbf{{#1}}}}
\newcommand{\NormalTok}[1]{{#1}}
\usepackage{graphicx,grffile}
\makeatletter
\def\maxwidth{\ifdim\Gin@nat@width>\linewidth\linewidth\else\Gin@nat@width\fi}
\def\maxheight{\ifdim\Gin@nat@height>\textheight\textheight\else\Gin@nat@height\fi}
\makeatother
% Scale images if necessary, so that they will not overflow the page
% margins by default, and it is still possible to overwrite the defaults
% using explicit options in \includegraphics[width, height, ...]{}
\setkeys{Gin}{width=\maxwidth,height=\maxheight,keepaspectratio}
\setlength{\parindent}{0pt}
\setlength{\parskip}{6pt plus 2pt minus 1pt}
\setlength{\emergencystretch}{3em}  % prevent overfull lines
\providecommand{\tightlist}{%
  \setlength{\itemsep}{0pt}\setlength{\parskip}{0pt}}
\setcounter{secnumdepth}{0}

%%% Use protect on footnotes to avoid problems with footnotes in titles
\let\rmarkdownfootnote\footnote%
\def\footnote{\protect\rmarkdownfootnote}

%%% Change title format to be more compact
\usepackage{titling}

% Create subtitle command for use in maketitle
\newcommand{\subtitle}[1]{
  \posttitle{
    \begin{center}\large#1\end{center}
    }
}

\setlength{\droptitle}{-2em}
  \title{}
  \pretitle{\vspace{\droptitle}}
  \posttitle{}
  \author{}
  \preauthor{}\postauthor{}
  \date{}
  \predate{}\postdate{}


% Redefines (sub)paragraphs to behave more like sections
\ifx\paragraph\undefined\else
\let\oldparagraph\paragraph
\renewcommand{\paragraph}[1]{\oldparagraph{#1}\mbox{}}
\fi
\ifx\subparagraph\undefined\else
\let\oldsubparagraph\subparagraph
\renewcommand{\subparagraph}[1]{\oldsubparagraph{#1}\mbox{}}
\fi

\usepackage{ctex} 
\setCJKmainfont{宋体}                     % 中文缺省字体,
\setCJKsansfont{黑体}                      % 中文无衬线字体,   调用命令: \sffamily
\setCJKmonofont{仿宋}     % 中文打字机(等宽)字体, 调用命令: \ttfamily

%\usepackage{ctex} 
%\setCJKmainfont{Adobe 宋体 Std}                     % 中文缺省字体,
%\setCJKsansfont{Adobe 黑体 Std}                      % 中文无衬线字体,   调用命令: \sffamily
%\setCJKmonofont{Adobe 仿宋 Std}     % 中文打字机(等宽)字体, 调用命令: \ttfamily

\begin{document}

\section{\texorpdfstring{\textbf{4.}}{4.}}\label{section}

\subsection{R code:}\label{r-code}

\subsection{1.读入初步整理的excel文件}\label{excel}

\begin{Shaded}
\begin{Highlighting}[]
\NormalTok{filename<-}\StringTok{"./advEconometric_homework_files/consumption_data.xlsx"}
\NormalTok{raw_data<-}\KeywordTok{read.xlsx}\NormalTok{(filename,}\DataTypeTok{sheetName =} \DecValTok{1}\NormalTok{,}\DataTypeTok{encoding =} \StringTok{"UTF-8"}\NormalTok{)}
\end{Highlighting}
\end{Shaded}

\subsection{\texorpdfstring{2.按``空格''符号分割表内数据及重命名列名}{2.按空格符号分割表内数据及重命名列名}}

\begin{Shaded}
\begin{Highlighting}[]
\NormalTok{col_name<-}\KeywordTok{names}\NormalTok{(raw_data)}
\NormalTok{data<-}\KeywordTok{cSplit}\NormalTok{(raw_data,col_name,}\StringTok{" "}\NormalTok{)}
\KeywordTok{names}\NormalTok{(data)<-}\KeywordTok{c}\NormalTok{(}\StringTok{"OBS"}\NormalTok{,}\StringTok{"year"}\NormalTok{,}\StringTok{"quarter"}\NormalTok{,}\StringTok{"YD"}\NormalTok{,}\StringTok{"CE"}\NormalTok{)}
\CommentTok{#y_q <- paste0(data$year,"-0",data$quarter)}
\CommentTok{#y_q <- as.character.Date(y_q)}
\end{Highlighting}
\end{Shaded}

\subsection{3.获取前一期数据}

\begin{Shaded}
\begin{Highlighting}[]
\NormalTok{log_c <-}\StringTok{ }\KeywordTok{log}\NormalTok{(data$CE)}
\NormalTok{log_c_1 <-}\StringTok{ }\KeywordTok{c}\NormalTok{()}
\NormalTok{log_c_1[}\DecValTok{2}\NormalTok{:}\DecValTok{200}\NormalTok{] <-}\StringTok{ }\NormalTok{log_c[}\DecValTok{1}\NormalTok{:}\DecValTok{199}\NormalTok{]}
\NormalTok{log_c_1[}\DecValTok{1}\NormalTok{] <-}\StringTok{ }\OtherTok{NA}
\NormalTok{log_y <-}\StringTok{ }\KeywordTok{log}\NormalTok{(data$YD)}
\NormalTok{log_y_1 <-}\StringTok{ }\KeywordTok{c}\NormalTok{()}
\NormalTok{log_y_1[}\DecValTok{2}\NormalTok{:}\DecValTok{200}\NormalTok{] <-}\StringTok{ }\NormalTok{log_y[}\DecValTok{1}\NormalTok{:}\DecValTok{199}\NormalTok{]}
\NormalTok{log_y_1[}\DecValTok{1}\NormalTok{] <-}\StringTok{ }\OtherTok{NA}

\NormalTok{data <-}\StringTok{ }\KeywordTok{cbind}\NormalTok{(data,log_c)}
\NormalTok{data <-}\StringTok{ }\KeywordTok{cbind}\NormalTok{(data,log_y)}
\NormalTok{data <-}\StringTok{ }\KeywordTok{cbind}\NormalTok{(data,log_c_1)}
\NormalTok{data <-}\StringTok{ }\KeywordTok{cbind}\NormalTok{(data,log_y_1)}
\end{Highlighting}
\end{Shaded}

\newpage

\subsection{4.获取差分数据}

\begin{Shaded}
\begin{Highlighting}[]
\NormalTok{delta_c <-}\StringTok{ }\KeywordTok{c}\NormalTok{()}
\NormalTok{delta_y <-}\StringTok{ }\KeywordTok{c}\NormalTok{()}
\NormalTok{delta_c[}\DecValTok{2}\NormalTok{:}\DecValTok{200}\NormalTok{] <-}\StringTok{ }\KeywordTok{diff}\NormalTok{(log_c)}
\NormalTok{delta_c[}\DecValTok{1}\NormalTok{] <-}\StringTok{ }\OtherTok{NA}
\NormalTok{delta_y[}\DecValTok{2}\NormalTok{:}\DecValTok{200}\NormalTok{] <-}\StringTok{ }\KeywordTok{diff}\NormalTok{(log_y)}
\NormalTok{delta_y[}\DecValTok{1}\NormalTok{] <-}\StringTok{ }\OtherTok{NA}
\NormalTok{data <-}\StringTok{ }\KeywordTok{cbind}\NormalTok{(data,delta_c)}
\NormalTok{data <-}\StringTok{ }\KeywordTok{cbind}\NormalTok{(data,delta_y)}
\CommentTok{#View(data)}
\end{Highlighting}
\end{Shaded}

\subsection{5.分别做三次回归}

\subsection{取1953-1996的数据}\label{1953-1996}

\begin{Shaded}
\begin{Highlighting}[]
\NormalTok{data_regression <-}\StringTok{ }\NormalTok{data[-}\KeywordTok{c}\NormalTok{(}\DecValTok{1}\NormalTok{:}\DecValTok{24}\NormalTok{),]}
\end{Highlighting}
\end{Shaded}

\subsection{对模型一进行调整 t-test}\label{-t-test}

\begin{Shaded}
\begin{Highlighting}[]
\NormalTok{fit1 <-}\StringTok{ }\KeywordTok{lm}\NormalTok{(log_c~log_c_1+log_y+delta_y,}\DataTypeTok{data =} \NormalTok{data_regression)}
\KeywordTok{summary}\NormalTok{(fit1)}
\end{Highlighting}
\end{Shaded}

\begin{verbatim}
## 
## Call:
## lm(formula = log_c ~ log_c_1 + log_y + delta_y, data = data_regression)
## 
## Residuals:
##       Min        1Q    Median        3Q       Max 
## -0.033875 -0.006364  0.000967  0.006110  0.025212 
## 
## Coefficients:
##             Estimate Std. Error t value Pr(>|t|)    
## (Intercept)  0.06394    0.02166   2.952   0.0036 ** 
## log_c_1      0.96923    0.02231  43.443  < 2e-16 ***
## log_y        0.02584    0.02105   1.227   0.2213    
## delta_y      0.26515    0.05645   4.697 5.37e-06 ***
## ---
## Signif. codes:  0 '***' 0.001 '**' 0.01 '*' 0.05 '.' 0.1 ' ' 1
## 
## Residual standard error: 0.0096 on 172 degrees of freedom
## Multiple R-squared:  0.9996, Adjusted R-squared:  0.9996 
## F-statistic: 1.633e+05 on 3 and 172 DF,  p-value: < 2.2e-16
\end{verbatim}

we can change the regressor so that gamma\_0+gamma\_1 is equal to the
coefficient of log\_y , t value = 1.227 not reject the null

\subsection{模型二的回归}

\begin{Shaded}
\begin{Highlighting}[]
\NormalTok{fit2 <-}\StringTok{ }\KeywordTok{lm}\NormalTok{(delta_c~log_c_1+delta_y+log_y_1,}\DataTypeTok{data =} \NormalTok{data_regression)}
\KeywordTok{summary}\NormalTok{(fit2)}
\end{Highlighting}
\end{Shaded}

\begin{verbatim}
## 
## Call:
## lm(formula = delta_c ~ log_c_1 + delta_y + log_y_1, data = data_regression)
## 
## Residuals:
##       Min        1Q    Median        3Q       Max 
## -0.033875 -0.006364  0.000967  0.006110  0.025212 
## 
## Coefficients:
##             Estimate Std. Error t value Pr(>|t|)    
## (Intercept)  0.06394    0.02166   2.952   0.0036 ** 
## log_c_1     -0.03077    0.02231  -1.379   0.1696    
## delta_y      0.29099    0.05511   5.280 3.86e-07 ***
## log_y_1      0.02584    0.02105   1.227   0.2213    
## ---
## Signif. codes:  0 '***' 0.001 '**' 0.01 '*' 0.05 '.' 0.1 ' ' 1
## 
## Residual standard error: 0.0096 on 172 degrees of freedom
## Multiple R-squared:  0.1913, Adjusted R-squared:  0.1772 
## F-statistic: 13.57 on 3 and 172 DF,  p-value: 5.499e-08
\end{verbatim}

\subsection{模型一的回归}

\begin{Shaded}
\begin{Highlighting}[]
\NormalTok{fit3 <-}\StringTok{ }\KeywordTok{lm}\NormalTok{(log_c~log_c_1+log_y+log_y_1,}\DataTypeTok{data =} \NormalTok{data_regression)}
\KeywordTok{summary}\NormalTok{(fit3)}
\end{Highlighting}
\end{Shaded}

\begin{verbatim}
## 
## Call:
## lm(formula = log_c ~ log_c_1 + log_y + log_y_1, data = data_regression)
## 
## Residuals:
##       Min        1Q    Median        3Q       Max 
## -0.033875 -0.006364  0.000967  0.006110  0.025212 
## 
## Coefficients:
##             Estimate Std. Error t value Pr(>|t|)    
## (Intercept)  0.06394    0.02166   2.952   0.0036 ** 
## log_c_1      0.96923    0.02231  43.443  < 2e-16 ***
## log_y        0.29099    0.05511   5.280 3.86e-07 ***
## log_y_1     -0.26515    0.05645  -4.697 5.37e-06 ***
## ---
## Signif. codes:  0 '***' 0.001 '**' 0.01 '*' 0.05 '.' 0.1 ' ' 1
## 
## Residual standard error: 0.0096 on 172 degrees of freedom
## Multiple R-squared:  0.9996, Adjusted R-squared:  0.9996 
## F-statistic: 1.633e+05 on 3 and 172 DF,  p-value: < 2.2e-16
\end{verbatim}

\end{document}
