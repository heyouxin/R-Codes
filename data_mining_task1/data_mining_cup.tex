\documentclass[]{article}
\usepackage{lmodern}
\usepackage{amssymb,amsmath}
\usepackage{ifxetex,ifluatex}
\usepackage{fixltx2e} % provides \textsubscript
\ifnum 0\ifxetex 1\fi\ifluatex 1\fi=0 % if pdftex
  \usepackage[T1]{fontenc}
  \usepackage[utf8]{inputenc}
\else % if luatex or xelatex
  \ifxetex
    \usepackage{mathspec}
  \else
    \usepackage{fontspec}
  \fi
  \defaultfontfeatures{Ligatures=TeX,Scale=MatchLowercase}
  \newcommand{\euro}{€}
\fi
% use upquote if available, for straight quotes in verbatim environments
\IfFileExists{upquote.sty}{\usepackage{upquote}}{}
% use microtype if available
\IfFileExists{microtype.sty}{%
\usepackage{microtype}
\UseMicrotypeSet[protrusion]{basicmath} % disable protrusion for tt fonts
}{}
\usepackage[margin=1in]{geometry}
\usepackage{hyperref}
\PassOptionsToPackage{usenames,dvipsnames}{color} % color is loaded by hyperref
\hypersetup{unicode=true,
            pdftitle={Data\_mining\_Cup 2017},
            pdfborder={0 0 0},
            breaklinks=true}
\urlstyle{same}  % don't use monospace font for urls
\usepackage{color}
\usepackage{fancyvrb}
\newcommand{\VerbBar}{|}
\newcommand{\VERB}{\Verb[commandchars=\\\{\}]}
\DefineVerbatimEnvironment{Highlighting}{Verbatim}{commandchars=\\\{\}}
% Add ',fontsize=\small' for more characters per line
\usepackage{framed}
\definecolor{shadecolor}{RGB}{248,248,248}
\newenvironment{Shaded}{\begin{snugshade}}{\end{snugshade}}
\newcommand{\KeywordTok}[1]{\textcolor[rgb]{0.13,0.29,0.53}{\textbf{{#1}}}}
\newcommand{\DataTypeTok}[1]{\textcolor[rgb]{0.13,0.29,0.53}{{#1}}}
\newcommand{\DecValTok}[1]{\textcolor[rgb]{0.00,0.00,0.81}{{#1}}}
\newcommand{\BaseNTok}[1]{\textcolor[rgb]{0.00,0.00,0.81}{{#1}}}
\newcommand{\FloatTok}[1]{\textcolor[rgb]{0.00,0.00,0.81}{{#1}}}
\newcommand{\ConstantTok}[1]{\textcolor[rgb]{0.00,0.00,0.00}{{#1}}}
\newcommand{\CharTok}[1]{\textcolor[rgb]{0.31,0.60,0.02}{{#1}}}
\newcommand{\SpecialCharTok}[1]{\textcolor[rgb]{0.00,0.00,0.00}{{#1}}}
\newcommand{\StringTok}[1]{\textcolor[rgb]{0.31,0.60,0.02}{{#1}}}
\newcommand{\VerbatimStringTok}[1]{\textcolor[rgb]{0.31,0.60,0.02}{{#1}}}
\newcommand{\SpecialStringTok}[1]{\textcolor[rgb]{0.31,0.60,0.02}{{#1}}}
\newcommand{\ImportTok}[1]{{#1}}
\newcommand{\CommentTok}[1]{\textcolor[rgb]{0.56,0.35,0.01}{\textit{{#1}}}}
\newcommand{\DocumentationTok}[1]{\textcolor[rgb]{0.56,0.35,0.01}{\textbf{\textit{{#1}}}}}
\newcommand{\AnnotationTok}[1]{\textcolor[rgb]{0.56,0.35,0.01}{\textbf{\textit{{#1}}}}}
\newcommand{\CommentVarTok}[1]{\textcolor[rgb]{0.56,0.35,0.01}{\textbf{\textit{{#1}}}}}
\newcommand{\OtherTok}[1]{\textcolor[rgb]{0.56,0.35,0.01}{{#1}}}
\newcommand{\FunctionTok}[1]{\textcolor[rgb]{0.00,0.00,0.00}{{#1}}}
\newcommand{\VariableTok}[1]{\textcolor[rgb]{0.00,0.00,0.00}{{#1}}}
\newcommand{\ControlFlowTok}[1]{\textcolor[rgb]{0.13,0.29,0.53}{\textbf{{#1}}}}
\newcommand{\OperatorTok}[1]{\textcolor[rgb]{0.81,0.36,0.00}{\textbf{{#1}}}}
\newcommand{\BuiltInTok}[1]{{#1}}
\newcommand{\ExtensionTok}[1]{{#1}}
\newcommand{\PreprocessorTok}[1]{\textcolor[rgb]{0.56,0.35,0.01}{\textit{{#1}}}}
\newcommand{\AttributeTok}[1]{\textcolor[rgb]{0.77,0.63,0.00}{{#1}}}
\newcommand{\RegionMarkerTok}[1]{{#1}}
\newcommand{\InformationTok}[1]{\textcolor[rgb]{0.56,0.35,0.01}{\textbf{\textit{{#1}}}}}
\newcommand{\WarningTok}[1]{\textcolor[rgb]{0.56,0.35,0.01}{\textbf{\textit{{#1}}}}}
\newcommand{\AlertTok}[1]{\textcolor[rgb]{0.94,0.16,0.16}{{#1}}}
\newcommand{\ErrorTok}[1]{\textcolor[rgb]{0.64,0.00,0.00}{\textbf{{#1}}}}
\newcommand{\NormalTok}[1]{{#1}}
\usepackage{graphicx,grffile}
\makeatletter
\def\maxwidth{\ifdim\Gin@nat@width>\linewidth\linewidth\else\Gin@nat@width\fi}
\def\maxheight{\ifdim\Gin@nat@height>\textheight\textheight\else\Gin@nat@height\fi}
\makeatother
% Scale images if necessary, so that they will not overflow the page
% margins by default, and it is still possible to overwrite the defaults
% using explicit options in \includegraphics[width, height, ...]{}
\setkeys{Gin}{width=\maxwidth,height=\maxheight,keepaspectratio}
\setlength{\parindent}{0pt}
\setlength{\parskip}{6pt plus 2pt minus 1pt}
\setlength{\emergencystretch}{3em}  % prevent overfull lines
\providecommand{\tightlist}{%
  \setlength{\itemsep}{0pt}\setlength{\parskip}{0pt}}
\setcounter{secnumdepth}{0}

%%% Use protect on footnotes to avoid problems with footnotes in titles
\let\rmarkdownfootnote\footnote%
\def\footnote{\protect\rmarkdownfootnote}

%%% Change title format to be more compact
\usepackage{titling}

% Create subtitle command for use in maketitle
\newcommand{\subtitle}[1]{
  \posttitle{
    \begin{center}\large#1\end{center}
    }
}

\setlength{\droptitle}{-2em}
  \title{Data\_mining\_Cup 2017}
  \pretitle{\vspace{\droptitle}\centering\huge}
  \posttitle{\par}
  \author{}
  \preauthor{}\postauthor{}
  \date{}
  \predate{}\postdate{}


% Redefines (sub)paragraphs to behave more like sections
\ifx\paragraph\undefined\else
\let\oldparagraph\paragraph
\renewcommand{\paragraph}[1]{\oldparagraph{#1}\mbox{}}
\fi
\ifx\subparagraph\undefined\else
\let\oldsubparagraph\subparagraph
\renewcommand{\subparagraph}[1]{\oldsubparagraph{#1}\mbox{}}
\fi

\usepackage{ctex} 
\setCJKmainfont{宋体}                     % 中文缺省字体,
\setCJKsansfont{黑体}                      % 中文无衬线字体,   调用命令: \sffamily
\setCJKmonofont{仿宋}     % 中文打字机(等宽)字体, 调用命令: \ttfamily

%\usepackage{ctex} 
%\setCJKmainfont{Adobe 宋体 Std}                     % 中文缺省字体,
%\setCJKsansfont{Adobe 黑体 Std}                      % 中文无衬线字体,   调用命令: \sffamily
%\setCJKmonofont{Adobe 仿宋 Std}     % 中文打字机(等宽)字体, 调用命令: \ttfamily

\begin{document}
\maketitle

\section{经济学院经济系 何友鑫 15320161152320}\label{--15320161152320}

\subsection{\texorpdfstring{经分析该问题比较适合用支持向量机进行分类,用R语言作为实验软件,用到的包有
``e1071''
``xlsx''}{经分析该问题比较适合用支持向量机进行分类,用R语言作为实验软件,用到的包有 e1071 xlsx}}\label{r-e1071-xlsx}

\subsection{读文件,确定自变量和因变量}

\begin{Shaded}
\begin{Highlighting}[]
\NormalTok{dataTrain <-}\StringTok{ }\KeywordTok{read.table}\NormalTok{(}\StringTok{"./竞赛实验数据2017/kddtrain2017.txt"}\NormalTok{)}
\NormalTok{x <-}\StringTok{ }\NormalTok{dataTrain[,-}\DecValTok{101}\NormalTok{]}
\NormalTok{y <-}\StringTok{ }\NormalTok{dataTrain[,}\DecValTok{101}\NormalTok{]}
\end{Highlighting}
\end{Shaded}

\subsection{计算SVM在2种分类机,4种核函数下模型的错误次数}\label{svm24}

\begin{Shaded}
\begin{Highlighting}[]
\NormalTok{type=}\KeywordTok{c}\NormalTok{(}\StringTok{"C-classification"}\NormalTok{,}\StringTok{"nu-classification"}\NormalTok{)}
\NormalTok{kernel=}\KeywordTok{c}\NormalTok{(}\StringTok{"linear"}\NormalTok{,}\StringTok{"polynomial"}\NormalTok{,}\StringTok{"radial"}\NormalTok{,}\StringTok{"sigmoid"}\NormalTok{)}
\NormalTok{accuracy=}\KeywordTok{matrix}\NormalTok{(}\DecValTok{0}\NormalTok{,}\DecValTok{2}\NormalTok{,}\DecValTok{4}\NormalTok{)}
\NormalTok{for (i in }\DecValTok{1}\NormalTok{:}\DecValTok{2}\NormalTok{)}
\NormalTok{\{}
  \NormalTok{for ( j in }\DecValTok{1}\NormalTok{:}\DecValTok{4}\NormalTok{) }
  \NormalTok{\{}
    \NormalTok{model <-}\StringTok{ }\KeywordTok{svm}\NormalTok{(x,y,}\DataTypeTok{type=}\NormalTok{type[i],}\DataTypeTok{kernel =} \NormalTok{kernel[j])}
    \NormalTok{pred_temp=}\KeywordTok{predict}\NormalTok{(model,x)}
    \NormalTok{accuracy[i,j]=}\KeywordTok{sum}\NormalTok{(pred_temp!=y)}
  \NormalTok{\}}
\NormalTok{\}}
\KeywordTok{dimnames}\NormalTok{(accuracy)=}\KeywordTok{list}\NormalTok{(type,kernel)}
\NormalTok{accuracy}
\end{Highlighting}
\end{Shaded}

\begin{verbatim}
##                   linear polynomial radial sigmoid
## C-classification    1212         16     31    2971
## nu-classification   1399        105    276    3229
\end{verbatim}

\begin{Shaded}
\begin{Highlighting}[]
\KeywordTok{print}\NormalTok{(}\KeywordTok{paste0}\NormalTok{(}\StringTok{"所有模型中最高的正确率为"}\NormalTok{,(}\DecValTok{6270-16}\NormalTok{)/}\DecValTok{6270}\NormalTok{))}
\end{Highlighting}
\end{Shaded}

\begin{verbatim}
## [1] "所有模型中最高的正确率为0.997448165869218"
\end{verbatim}

\subsection{\texorpdfstring{由以上结果可知,使用SVM进行实验,type=``C-classification'',kernel
=
``polynomial''的模型最优。}{由以上结果可知,使用SVM进行实验,type=C-classification,kernel = polynomial的模型最优。}}\label{svmtypec-classificationkernel-polynomial}

\subsection{实验1用训练数据的前5770条作为训练集,后500条作为测试集,看看预测结果}\label{15770500}

\begin{Shaded}
\begin{Highlighting}[]
\NormalTok{model1 <-}\StringTok{ }\KeywordTok{svm}\NormalTok{(x[}\DecValTok{1}\NormalTok{:}\DecValTok{5770}\NormalTok{,],y[}\DecValTok{1}\NormalTok{:}\DecValTok{5770}\NormalTok{],}\DataTypeTok{type=}\StringTok{"C-classification"}\NormalTok{,}\DataTypeTok{kernel =} \StringTok{"polynomial"}\NormalTok{)}
\NormalTok{pred1 <-}\StringTok{ }\KeywordTok{predict}\NormalTok{(model1,x[}\DecValTok{5771}\NormalTok{:}\DecValTok{6270}\NormalTok{,])}
\end{Highlighting}
\end{Shaded}

\begin{Shaded}
\begin{Highlighting}[]
\KeywordTok{table}\NormalTok{(pred1,y[}\DecValTok{5771}\NormalTok{:}\DecValTok{6270}\NormalTok{])}
\end{Highlighting}
\end{Shaded}

\begin{verbatim}
##      
## pred1   0   1   2
##     0 134   1   9
##     1   2 142   2
##     2  13   8 189
\end{verbatim}

\subsection{实验2用训练数据的前6000条作为训练集,后270条作为测试集,看看预测结果}\label{26000270}

\begin{Shaded}
\begin{Highlighting}[]
\NormalTok{model2 <-}\StringTok{ }\KeywordTok{svm}\NormalTok{(x[}\DecValTok{1}\NormalTok{:}\DecValTok{6000}\NormalTok{,],y[}\DecValTok{1}\NormalTok{:}\DecValTok{6000}\NormalTok{],}\DataTypeTok{type=}\StringTok{"C-classification"}\NormalTok{,}\DataTypeTok{kernel =} \StringTok{"polynomial"}\NormalTok{)}
\NormalTok{pred2 <-}\StringTok{ }\KeywordTok{predict}\NormalTok{(model2,x[}\DecValTok{6001}\NormalTok{:}\DecValTok{6270}\NormalTok{,])}
\end{Highlighting}
\end{Shaded}

\begin{Shaded}
\begin{Highlighting}[]
\KeywordTok{table}\NormalTok{(pred2,y[}\DecValTok{6001}\NormalTok{:}\DecValTok{6270}\NormalTok{])}
\end{Highlighting}
\end{Shaded}

\begin{verbatim}
##      
## pred2  0  1  2
##     0 73  0  3
##     1  0 84  0
##     2  7  5 98
\end{verbatim}

\subsection{实验3使用全部训练样本展示预测结果,并与真实情况的比较。}\label{3}

\begin{Shaded}
\begin{Highlighting}[]
\NormalTok{model_fitted <-}\StringTok{ }\KeywordTok{svm}\NormalTok{(x,y,}\DataTypeTok{type=}\StringTok{"C-classification"}\NormalTok{,}\DataTypeTok{kernel =} \StringTok{"polynomial"}\NormalTok{)}
\KeywordTok{summary}\NormalTok{(model_fitted)}
\end{Highlighting}
\end{Shaded}

\begin{verbatim}
## 
## Call:
## svm.default(x = x, y = y, type = "C-classification", kernel = "polynomial")
## 
## 
## Parameters:
##    SVM-Type:  C-classification 
##  SVM-Kernel:  polynomial 
##        cost:  1 
##      degree:  3 
##       gamma:  0.01 
##      coef.0:  0 
## 
## Number of Support Vectors:  5115
## 
##  ( 1443 2101 1571 )
## 
## 
## Number of Classes:  3 
## 
## Levels: 
##  0 1 2
\end{verbatim}

\begin{Shaded}
\begin{Highlighting}[]
\NormalTok{pred <-}\StringTok{ }\KeywordTok{predict}\NormalTok{(model_fitted,x)}
\end{Highlighting}
\end{Shaded}

\begin{Shaded}
\begin{Highlighting}[]
\KeywordTok{table}\NormalTok{(pred,y)}
\end{Highlighting}
\end{Shaded}

\begin{verbatim}
##     y
## pred    0    1    2
##    0 1963    1    0
##    1    6 1839    1
##    2    7    1 2452
\end{verbatim}

\subsection{由实验的的结果来看,模型是可信的。}

\subsection{由于自变量有100个,所以不好进行权重优化,99.74\%的准确率是可以接受的范围,故即用原始模型作为最终模型}\label{10099.74}

\subsection{读测试数据,并用模型进行预测,将结果写入excel文件中}\label{excel}

\begin{Shaded}
\begin{Highlighting}[]
\NormalTok{dataTest <-}\StringTok{ }\KeywordTok{read.table}\NormalTok{(}\StringTok{"./竞赛实验数据2017/kddtest2017.txt"}\NormalTok{)}
\NormalTok{pred_test=}\KeywordTok{predict}\NormalTok{(model_fitted,dataTest)}
\KeywordTok{write.xlsx}\NormalTok{(pred_test,}\StringTok{"predict_result.xlsx"}\NormalTok{,}\DataTypeTok{col.names =} \NormalTok{F,}\DataTypeTok{row.names =} \NormalTok{F)}
\end{Highlighting}
\end{Shaded}

\end{document}
